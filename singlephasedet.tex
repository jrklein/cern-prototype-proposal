\subsection{LBNF detector}
Description of LBNF far detector components
\begin{itemize}
\item Overview of the far detector option
\begin{itemize}
\item List major components
\end{itemize}
\item Dimensions
\item Need for Modularity
\item Scaling from previous detectors
\item Possible development paths
\item Parameter summary
\end{itemize}


\subsection{CERN prototype detector}
Detailed description of CERN prototype detector components
%\begin{itemize}
\subsubsection{Overview of the CERN test Detector}

\textbf{TPC description and size (from Jack Fowler)}

The TPC will be assembled from elements that are of the same size as those planned for the single phase 
far detector.  The overall size of the TPC will be derived by the size and number of anode planes (APA).  It 
has been determined in order to perform the required physics, the TPC will consist of three APAs.  The 
APAs will have an active area 2.29 m wide and 6.0 m high.  These active area dimensions result in an APA 
that is 2.32 m wide and 6.29 m high.  The combination of the three APAs will determine the overall length 
of the TPC.  This is 7.2 m.  There will be a cathode plane (CPA) on either side of the APAs.  The size of the 
CPAs is determined by the active area of the three APAs.  The active area of the three APAs is 
approximately 7.2 m wide by 6.2 m high.  The drift distance between the CPAs and row of APAs will be 2.5 
m.  The overall width of the TPC will be determined by a combination of the drift distances along with the 
thickness of the APA, which is constructed of 3” x 4” stainless steel (SS) structural tubing.  The overall 
width of the TPC is 5.2 m.  Like the length of the TPC, the overall height will be determined by the height 
of the APA.  The overall height of the TPC will be 6.3 m.  The TPC dimensions will be 7.2 m long x 5.2 m 
wide x 6.3 m high.

Along with the APAs and CPAs, the TPC will include a field cage that surrounds the entire assembly.  It will 
be designed similarly to the field cage in phase 2 of the 35t experiment at FNAL.  This is a series of 
protruded fiberglass I beams for the structural elements.  These I-beams will be tiled with large copper 
sided FR4 panels to create the field cage.  Each panel will be connected with a series of resistors.  The field 
cage will be connected to the CPAs through a capacitor assembly.

All of this will be supported by rows of I-beams supported from a mechanical structure above the cryostat.  
The hangers for these I-beams will pass through the insulated top cap.  There will be a series of feed thru 
flanges in the top cap of the cryostat to bring in and take out services for the TPC.  There will be a HV feed 
thru for each of the CPAs and one signal feed thru for each of the APAs

\textbf{Cryostat size from TPC dimensions  (from Jack Fowler)}

The minimum internal size of the cryostat is determined from size of the TPC.  At the bottom of the 
cryostat there needs to be a minimum of 0.3 m between the frame of the CPA and closest point on the SS 
membrane.  This is to prevent high voltage discharge between the CPA and the electrically grounded 
membrane. It is foreseen that there would be some cryogenic piping and instrumentation under the TPC.  
There is a height allowance of 0.1 m for this.  There will be access and egress space around the outside 
of the TPC and the membrane walls.  On three sides, 1.0 m of space is reserved for this.  The final side of 
the TPC will have piping and instrumentation for the cryogenic system.  There will be 1.3 m of space 
reserved for this.  

The support system for the TPC will be located at the top between the underside of the cryostat roof and 
the top of the TPC.  The plan is to model this space similar to what is planned for the far site TPC.  There 
will be 0.9 m of ullage space.  In order to prevent high voltage discharge, the upper most part of the CPA 
needs to be submerged a minimum of 0.3 m below the liquid Argon surface.  The top of the TPC will be 
separated from the membrane by a minimum of 1.2 m.  

Adding all of these to the size of the TPC yields the minimum inner dimensions of the cryostat.  A 
minimally sized cryostat would be 9.5 m long, 7.3 m wide and 8.4 m high.  This assumes the TPC will be 
positioned inside the cryostat with the CPAs and end field cages parallel to the walls of the cryostat.  Also 
there is no space allotted for a beam window to enter the cryostat.  Clearance would need to be added if 
it violates any of the current boundaries listed above.  
These dimensions also preserve the ability to reverse the order of the APAs and CPAs inside the TPC.  The 
current plan is to have the APAs located in the center of the cryostat with a CPA on each side.  Reversing 
this to have the CPA in the center and APAs on each side may be required to achieve some of the 
proposed physics.  The orientation of the TPC components will be finalized after various scenarios have 
been sufficiently simulated.  


\subsubsection{Parameters table}
\subsubsection{Requirements (data rate, dimensions, gap to wall,  ?)}
\subsubsection{Installation}

\textbf{Installation Plans for the TPC into the Cryostat  (from Jack Fowler)}

The interior of the cryostat will be prepared prior to the installation of the TPC.  A series of support rails 
will be suspended below the top surface of the cryostat membrane.  These will be structurally supported 
by a truss structure above the cryostat.  These supports will pass through the top of the cryostat.  They 
need to be designed to minimize the heat gain into the cryogenic volume.  For the CPAs, the rails need to 
be electrically isolated due to high voltage concerns.  To preserve the ability to reverse the order of the 
TPC components, all of the support rails will be designed to the same set of requirements regarding 
loads and attachment points.  

There will be a series of feed thru flanges located along each of the support rails.  These will be cryogenic 
flanges where the services for the TPC components can pass through the top of the cryostat.  It is 
foreseen that each CPA will require one feed thru for the high voltage probe to bring in the drift voltage.  
The drift voltage is 500 V/cm.  For a drift distance of 2.5 m, the probe voltage will be 125 kV.  There will 
be one service feed thru for each of the APAs.  These feed thrus will include high speed data, bias 
voltages for the wire planes, control and power for the cold electronics.  

The main TPC components will be installed through a large hatch in the top of the cryostat.  This is 
similar to the installation method intended for the detector at the far site.  This hatch will have an 
aperture approximately 2.0 m wide and 3.5 m long.  Each APA and CPA panel will be carefully tested after 
transport into the clean area and before installation into one of the cryostats. Immediately after a panel is 
installed it will be rechecked. The serial installation of the APAs along the rails means that removing and 
replacing one of the early panels in the row after others are installed would be very costly in effort and 
time. Therefore, to minimize the risk of damage, as much work around already installed panels as 
possible will be completed before proceeding with further panels.
The installation sequence is planned to proceed as follows:

\begin{enumerate}
\item Install the monorail or crane in the staging area outside the cryostat, near the equipment hatch.
\item Install the relay racks on the top of the cryostat and load with the DAQ and power supply crates.
\item Dress cables from the DAQ on the top of the cryostat to remote racks.
\item Construct the clean-room enclosure outside the cryostat hatch.
\item Install the raised-panel floor inside the cryostat. 
\item Insert and assemble the stair tower and scaffolding in the cryostat.
\item Install the staging platform at the hatch entrance into the cryostat.
\item Install protection on (or remove) existing cryogenics instrumentation in the cryostat.
\item Install the cryostat feedthroughs and dress cables inside the cryostat along the support beams.
\item Install TPC panels:
   \begin{enumerate}
   \item Install both CPA panels.  These will be installed from the floor of the cryostat.  Access to the top edge will be required by scaffolding.  
   \item Install and connect HV probe for each of the CPAs.
   \item Perform electrical tests on the connectivity of the probe to the CPAs.
   \item Install first end wall of vertical field cage at the non-access end of the cryostat.  These will be installed from the floor of the cryostat.  Scaffolding will be needed to install the supporting structure and then attach the panels to the structure.  
   \item Test the inner connections of the field cage panels.
   \item Install the first APA and connect to the far end field cage support.
   \item Connect power and signal cables.  This will require scaffolding to access the top edge of the APA.
   \item Test each APA wire for expected electronics noise. Spot-check electronics noise while cryogenics equipment is operating.
   \item Install the upper field cage panels for the first APA between the APA and CPAs  This will require scaffolding to access the upper edge of the APA, CPA and field cage structure.
   \item Perform electrical tests on upper field cage panels.
   \item Repeat steps (f) through (j) for the next two APAs.
   \item Install the lower field cage panels between the APAs and CPAs.  Start at the far end away from the access hatch and work towards the hatch. 
   \item Perform electrical test on lower field cage panels and the entire loop around the TPC.
   \item Remove temporary floor sections as the TPC installation progresses.
   \item Install sections of argon-distribution piping as the TPC installation progresses.
   \item Install the final end wall of vertical field cage at the access end of the cryostat.  These will be installed from the floor of the cryostat.  Scaffolding will be needed to install the supporting structure and then attach the panels to the structure.
   \end{enumerate}
\item Remove movable scaffold and stair towers.
\item Temporarily seal the cryostat and test all channels for expected electronics noise.
\item Seal the access hatch.
\item Perform final test of all channels for expected electronics noise.
\end{enumerate}
 
 In general, APA panels will be installed in order starting with the panel furthest from the hatch side of 
 the cryostat and progressing back towards the hatch. The upper field cage will be installed in stages as 
 the installation of APA and CPAs progresses.  After the APAs are attached to the support rods the 
 electrical connections will be made to electrical cables that were already dressed to the support beams 
 and electrical testing will begin. Periodic electrical testing will continue to assure that nothing gets 
 damaged during the additional work around the installed APAs.  

  The TPC installation will be performed in three stages, each in a separate location; the locations, or 
  zones, are shown in Figure x-xx (this illustration was made for a 34-kton, in-line underground 
  detector, but the work zones are also applicable for the 10-kton surface siting). First, in the clean room 
  vestibule, a crew will move the APA and CPA panels from storage racks, rotate to the vertical position 
  and move them into the cryostat. Secondly, in the panel-staging area immediately below the equipment 
  hatch of the cryostat, a second crew will transfer the lower panels from the crane to the staging 
  platform, connect the upper and lower panels together, route cables to the top of stacked panels and f
  inally transfer the stacked panels on to the monorail trolley that moves within the cryostat. A third crew 
  will reposition the movable scaffolding and use the scaffold to make the mechanical and electrical 
  connections at the top for each APA and CPA as they are moved into position. The monorails inside and 
  outside the cryostat will each have two motorized trolleys so that work can be conducted by all three 
  crews in parallel. The steady-state rate for installation, given this work plan and a single-shift schedule, 
  is estimated to be two stacked panels per day.  

The requirements for alignment and survey of the TPC are under development. Since there will be plenty 
of cosmic rays in the surface detector and beam events, significant corrections can be made for any 
misalignment of the TPC. The current plan includes using a laser guide or optical transit and the 
adjustment features of the support rods for the TPC to align the top edges of the APAs in the TPC to be 
straight, level and parallel within a few mm. The alignment of the TPC in other dimensions will depend on 
the internal connecting features of the TPC.  The timing of the survey will depend on understanding when 
during the installation process the hanging TPC elements are in a dimensionally stable state. The 
required accuracy of the survey is not expected to be finer than a few mm.  
%\end{itemize}

\subsubsection{APA}
\begin{itemize}
\item General description of the APA
\item Justification for the basic design (Dimensions, wire Wrapping, wire pitch and angle, hung nature)
\item Description of construction technique
\end{itemize}

\subsubsection{CPA and Field Cage}
\begin{itemize}
\item General description of the CPA - material, HV coupling, Stored energy, ….
\item Description of construction and installation
\item General description of the field cage
\item Description of design alternatives and engineering plan
\item Overview of design and installation plan
\end{itemize}

\subsubsection{TPC Readout}


The TPC front-end (FE) electronics  operate at cryogenic temperatures. 
The system provides amplification, shaping,  digitization, buffering and
multiplexing of the signals. The FE electronics consist of three boards stacked on top of one another:  the Analog Mother Board, the FPGA Mezzanine Board, and the SERDES Mezzanine Board.  Figure \ref{fig:coldelec} shows a schematic of the cold FE electronics. 

\begin{figure}[tb]
  \centering
\includegraphics[scale=0.4]{figures/fe-electronics-block-diagram.pdf}
\label{fig:coldelec}
  \caption{Schematic for the TPC cold FE electronics. ***MG*** Not sure if this is high enough quality.  ***Ask Chen for source and fix numbers.  }
\end{figure}

The Analog Mother Board contains the front-end ASIC chips which perform the analog readout of the TPC wires.  The FE ASIC chip is implemented as a
mixed-signal ASIC providing amplification, shaping, digitization,
buffering, a 16:2 multiplexing stage, a driver and voltage regulators.  The analog-to-digital converter on the ASIC samples each TPC wire at 2 MHz.  Eight such chips 
are mounted on a single readout board, instrumenting 128 adjacent wires in one plane. 

The two (multiplexed) signals from each FE ASIC are fed into the FPGA Mezzanine Board.  The cold FPGA aggregates the TPC data and also supplies the control and clock to the FE ASICs.   The FPGA on the mezzanine board receives the data and packages the 128 channels together, one 2 MHz clock tick at a time.  This is then sent to the SERDES board for serialization and sent to the cryostat flange board over high-speed (1 Gbps) serial links and finally to the DAQ system.  

Besides the high-speed signal cable, which is a twin-axial cable bundle manufactured by GORE, there are cable bundles for low-voltage power, wire-bias voltages, and various slow controls and monitoring. Redundant cables will be provided for many of these functions. The  cable bundles will be connected through a feedthrough on the roof of the cryostat. 


\begin{tabular}{l|c}
\hline
 parameter & value \\
  \hline
 ADC Sampling Rate & 2 MHz\\
More stuff   & \\
 Cluster-on-Boards (COB) & 3\\
Data-Processing-Modules (DPM) & 12 \\
ATCA Shelves  & 1 (6-slot)\\
TPC Readout Compute Nodes & 3 \\
\hline
\end{tabular}



The primary interface between the TPC front-end electronics (FE) and the DAQ subsystem consists of an ATCA-based system of RCEs (Reconfigurable Cluster Elements).  The RCE system receives the serialized raw data for the FE, performs zero-suppression on it, and packetizes and transmits the resulting sparsified data to a back-end data farm for event building and further processing.  Additionally, the RCE system transmits timing and control signals to the FE as well as forwarding configuration data to them at start-up.     

The RCE system consists the following components:  a commercial ATCA shelf (2-, 6-, or 14-slot), a Cluster-On-Board (COB) which is the "front board" in ATCA terms, and a Rear-Transition-Module (RTM) which is the "rear board". A schematic of the system is shown in Figure \ref{fig:rceblock}.  The COB is a custom board, developed by SLAC, which holds the processing power of the system.  The COB (see Figure \ref{cob}) consists of 5 bays for holding daughter boards, an onboard 10-GbE switch, and both 10- and 1-Gb ethernet connections for communications with the back-end system.  Four of the daughter-board bays are for Data Processing Modules (DPM), each of which can hold up to two RCEs.  The RCE is the core procession unit of the system; it is made up of a modern SoC (currently, the Xilinx Zynq-7045) with multiple high-speed I/O ports (up to 10-Gbps each) and external DRAM and flash memory controllers.  The other bay on the COB contains the Data Transmission Module (DTM) which is responsible  for distributing timing and trigger information to and between the DPMs.  

While the COB hardware is application agnostic,  the RTM is application specific. The RTM provides the mechanical interface between the front-end (or, in our case, the flange electronics) and the back-end, as well as other external sources such as the timing or trigger systems.  In this case we will use fiber optic connections between the flange and the TPC DAQ using 8 12-channel (full duplex) CXP connectors on the RTM. 

With the assumption that each cold FE board multiplexes it's 128 wire channels to 4 outputs at 1-Gbps each, the non-zero suppressed data for 1 APA can be fed into a single COB (containing 8 RCEs).  Each RCE would receive data from 2 FE boards, perform zero-suppression, and send the result to the back-end.  

MG***data rates?  

\begin{figure}[tb]
  \centering
%\includegraphics[scale=0.4]{figures/cern-tpc-daq-block.pdf}
\includegraphics[scale=0.4]{figures/rce-block.pdf}
\label{fig:rceblock}
  \caption{Schematic for the TPC DAQ system.   }
\end{figure}


\begin{figure}[hbt]
  \includegraphics[scale=0.6]{figures/COB-gen3.pdf}
  \caption{\label{fig:cob} The COB (left of the large connectors) and RTM (right).  }
\end{figure}




\begin{itemize}
\item Overview of the TPC readout chain (include main parameters)
\item Overview of the cold electronics
\item Overview of the RCE and interface to the DAQ
\item	Discussion of sparsification and triggering
\end{itemize}

\subsubsection{Photon Detection System}
\begin{itemize}
\item General description of the photon system including requirements
\item Overview of the photon system alternatives and selection process
\item Description of the readout and development plans
\end{itemize}

\subsubsection{DAQ, Slow control and monitoring}
\include{DAQ}
\begin{itemize}
\item General description of the DAQ system
\item Plans and status for the 35t and next steps
\item Needs for a slow control system
\end{itemize}

\subsubsection{Offline requirements and software}



