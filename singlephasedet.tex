
\subsection{LBNF detector}

The far detector for the ELBNF collaboration will be a series of four 17 kt (10 kt) active (fiducial) volume  liquid argon time projection chambers instrumented with photon detection. It is planned that the first 10 kt detector will be ready for installation in the 2021 timeframe. One option for the TPC design is a wire plane based TPC with cold electronics readout. Designs of this style are are referred to as single-phase detectors as the charge generation, drift, and detection all occurs in the argon liquid phase. This style TPC has the advantage that there is no charge amplification before collection making a very precise charge measurement possible. To achieve ELBNF's goals, a detector much larger than ICARUS, the largest LAr TPC detector built to date, is needed. The LBNE experiment was developing a scalable far detector design shown in Figure \ref{fig:fardet-overview} that would scale-up LAr TPC technology by roughly a factor of 40 compared to the ICARUS T600 detector. To achieve this scale-up, a number of novel design elements need to be employed. A membrane cryostat typical for the liquefied natural gas industry will be used instead of a conventional evacuated cryostat. The wire planes or anode plane assemblies (APAs) will be factory-built as planar modules that are then installed into the cryostat. The modular nature of the APAs allow the size of the detector to be scaled up to at least 40 kt fiducial mass. Both the analog and digital electronics will be mounted on the wire planes inside the cryostat in order to reduce the electronic noise, to avoid transporting analog signals large distances, and to reduce the number of cables that penetrate the cryostat. The scintillation photon detectors will employ light collection paddles to reduce the required photo-cathode area. Many of the aspects of the design will be tested in a small scale prototype at Fermilab but given the very large scale of the detector elements a full-scale test is highly desirable. As the new ELBNF collaboration forms and organized a combined detector design team will emerge. Ideas from this new collaboration will modify the design presented here but this designs provides a concrete example of a possible future detector. 



\begin{figure}[!htb]
\centering
\begin{minipage}[b]{1.0\textwidth}
\begin{center}
\includegraphics[width=.75\textwidth]{./figures/fardet-3D.png}
%\includegraphics[width=0.7\textwidth]{EndView-sketch.png}
\end{center}
\end{minipage}
\caption{\small 3D model of one design of the ELBNF single-phase detector. Shown is 5kt fiducial volume detector which would need to be lengthened for the 10 kt design. The present ELBNF plan calls for the construction of 4 10 kt detectors of similar design. }
\label{fig:fardet-overview} 
\end{figure}

The goals of the ELBNE detector test can be broken into four categories: argon contamination mitigation verification, TPC mechanical verification, TPC electrical verification, and photon detection light yield verification. Research at Fermilab utilizing the Materials Test Stand has shown that electronegative contamination to the ultra-pure argon from all materials tested is negligible if the material is under the liquid argon. This implies that the dominant source of contamination originates from the gas ullage region and in the room temperature connections to the detector. Careful design of the ullage region to insure that all surfaces and feedthroughs are cold is expected to greatly reduce the sources of contamination over what exists in present detectors. Other concepts attempt to eliminate the gas ullage completely. The goals related to mechanical testing are to test the integrity of the detector. In the current design, each APA measures 2.3 m by 6.0 m and includes 2560 wires and associated readout channels. Given the complexity of these assemblies, a test where the detector can be thermally cycled and tested under operating conditions is highly advised prior to mass production. The mechanical support of the APAs can be tested to verify that the mechanical design is reliable and will accommodate any necessary motion between the large wire planes. The impact of vibration isolation between the cryostat roof and the detector can also be tested. Finally a potential improvement in the cryostat design is the possibility to move the pumps external to the main cryostat. This will reduce any mechanical coupling to the detector and also greatly improve both reliability and ease of repair. The electrical testing goals are to insure that the high voltage design is robust and that the required low electronic noise level can be achieved. As the detector scale increases so does the capacitance and the stored energy in the device. The design of the field cage and high voltage cathode planes needs to be such that HV discharge is unlikely and that if the event occurs no damage to the detector or cryostat results. The grounding and shielding of large detectors is also critical for low noise operation. By testing the full scale elements one insures that the grounding plan is fully developed and effective. Large scale tests of the resulting design will verify the electrical model of the detector. 


\subsection{CERN prototype detector}

\subsubsection{Overview of the CERN Single-Phase test Detector}
\begin{figure}[htb]
\centering
\begin{minipage}[b]{1.0\textwidth}
\begin{center}
\includegraphics[width=.75\textwidth]{./figures/TPC-3D-section.jpg}
\end{center}
\end{minipage}
\caption{\small 3D model of the single-phase detector prototype is shown inside the test cryostat.}
\label{fig:CERNdet-overview}
\end{figure}

This sections presents the design details of a single-phase detector based on the development of the LBNE collaboration. As ELBNF moves forward the TPC working group will evaluate this and any modifications or alternate proposals. For the purpose of this proposal this represents one alternate, and it is expected to evolve as the new collaboration organized and more work is done. 

This TPC consists of alternating anode plane assemblies (APAs) and cathode plane assemblies (CPAs), with field-cage panels enclosing the four open sides between the anode and cathode planes.  Figure  \ref{fig:CERNdet-overview} shows a sectioned view for the planned TPC inside the cryostat at CERN.  A uniform electric field is created in the volume between the anode and cathode planes. A charged particle traversing this volume leaves a trail of ionization. The electrons drift toward the anode plane, which is constructed from multiple layers of sense wires, inducing electric current signals in the front-end electronic circuits connected to the wires.

To the extent possible TPC will be assembled from elements that are of the same size as those planned for the single phase far detector.  The primary exception to this is the length of the field cage panels which are 2.5m in this design, compared to 3.4m in the far detector. This is to reduce the impact of space charge on the prototype necessitated by the surface operation. The overall size of the TPC will be derived by the size and number of anode planes (APA).  It has been determined in order to perform the required physics, the TPC will have a 3 APA wide active volume.  The APAs will have an active (total) area 2.29 m (2.32 m) wide and 6.0 m (6.2 m) high . The combination of the three APAs will determine the overall TPC length to be 7.2 m. There will be a cathode plane (CPA) on either side of the APAs.  The overall width of the TPC will be determined by a combination of the drift distances along with the thickness of the APA, which is constructed of 76.2 x 101.6 mm stainless steel (SS) structural tubing.  The overall width of the TPC is 5.2 m.  Like the length of the TPC, the overall height will be determined by the height of the APA which is 6.3 m.  In summary the external TPC dimensions will be 7.2 m long x 5.2 m wide x 6.3 m high. Along with the APAs and CPAs, the TPC will include a field cage that surrounds the entire assembly.  This is a series of pultruded fiberglass I beams for the structural elements.  These I-beams will be tiled with large copper sided FR4 panels to create the field cage.  Each panel will be connected with a series of resistors.  The field cage will be connected to the CPAs through a capacitor assembly.

All of this will be supported by rows of I-beams supported from a mechanical structure above the cryostat.  The hangers for these I-beams will pass through the insulated top cap.  There will be a series of feed thru flanges in the top cap of the cryostat to bring in and take out services for the TPC.  There will be a HV feed thru for each of the CPAs and one signal feed thru for each of the APAs.


The minimum internal size of the cryostat is 9.5 m long, 7.3 m wide and 8.4 m high.  This is determined from size of the TPC.  At the bottom of the 
cryostat there needs to be a minimum of 0.3 m between the frame of the CPA and closest point on the SS membrane.  This is to prevent high voltage discharge between the CPA and the electrically grounded membrane. It is foreseen that there would be some cryogenic piping and instrumentation under the TPC.  There is a height allowance of 0.1 m for this.  There will be access and egress space around the outside of the TPC and the membrane walls.  On three sides, 1.0 m of space is reserved for this.  The final side of the TPC will have piping and instrumentation for the cryogenic system.  There will be 1.3 m of space reserved for this.  

The support system for the TPC will be located at the top between the underside of the cryostat roof and  the top of the TPC.  The plan is to model this space similar to what is planned for the far site TPC.  There will be 0.9 m of ullage space.  In order to prevent high voltage discharge, the upper most part of the CPA needs to be submerged a minimum of 0.3 m below the liquid Argon surface.  The top of the TPC will be separated from the membrane by a minimum of 1.2 m.  

These dimensions also preserve the ability to reverse the order of the APAs and CPAs inside the TPC.  The current plan is to have the APAs located in the center of the cryostat with a CPA on each side.  Reversing this to have the CPA in the center and APAs on each side may be required to minimize the dead space between the two drift volumes.    


% File from Bo Yu on the TPC component design
% Input from Bo and Lee Greenler on the TPC detail design

\subsubsection{Anode Plane Assemblies (APAs)}

% input from Lee Greenler



\begin{figure}[!htb]
\centering
\begin{minipage}[b]{1.0\textwidth}
\begin{center}
\includegraphics[width=.75\textwidth]{./figures/TPC_APA_1}
\includegraphics[width=0.75\textwidth]{./figures/TPC_APA_2}
\end{center}
\end{minipage}
\caption{Clockwise from left: A full size APA, an APA corner showing the electronics boards, an APA lower corner photo showing wires and edge boards, and a figure showing the wire orientations and the placement of wire aligning combs. }
\label{fig:tpc_apa_overview} 
\end{figure}


Each APA (Figure \ref{fig:tpc_apa_overview}) is instrumented with 3 layers of signal wires, one longitudinal collection plane and two 37.5$^\circ$ angled induction planes with an additional outer grid plane that helps maintain the field. The overall dimensions of the active area as mentioned able are 2.3 m wide, 6 m long. The dimension of the wire planes were selected to fit down the Ross shaft at SURF, be compatible with a standard HiCube transport container, and allow construction from readily available materials.  The angled layers start at the electronics end and wind around to the other side on their way to the bottom. The wire angle was selected so that a given angled induction wire will not overlay any longitudinal collection wire more than once in order to reduce ambiguities caused by the wrapped wire construction. Partial wire layers are shown here in Figure ~\ref{fig:tpc_apa_overview} at the bottom.  With a wire pitches of 4.67 mm (diagonal layers) and 4.79 (straight layers), the total number of readout channels in an APA is 2560.
The grid layer is not depicted in Figure ~\ref{fig:tpc_apa_overview} for clarity. The underlying structure of each APA is a framework of rectangular, stainless steel tubing.  The side and bottom edges of the frame are lined with multiple layers of fiberglass circuit boards, notched along the edges to support and locate the wires that cross the APA face. A set of FR4 combs are glued to the APA frame to capture the wires at regular intervals. The front-end electronics boards are mounted at the top end of the frame and protected by a metal enclosure. 


The distance between wire planes is 4.8 mm (3/16 in) corresponding with standard printed circuit board thickness, and while maintaining optimal signal formation. The four wire planes will be electrically biased so that electrons from an ionizing-particle track completely drift past the first three planes and are collected by the fourth plane. Calculations show that the minimum bias voltages needed to achieve this goal are $V_G$ = -665V, $V_U$ = -370V, $V_V$ = 0V and $V_X$ = 820V respectively (where G, U, V, and X are the wire-layer labels from outside in, towards the frame).  It is convenient to set one of the wire planes to ground so that the wires can be DC coupled to the front-end readout electronics. In this instance, the V wire plane is set to ground potential to reduce the maximum bias voltages on the other wire planes, and enable the use of lower voltage rated AC coupling capacitors. A grounded mesh plane, located 4.8 mm behind the collection (X) plane, prevents the electric field around this set of wires from being distorted by the metal frame structure and the wires on the opposite side of the frame. It also shields the sensing wires from potential EM interferences from the photon detectors (Fig.~\ref{fig:pd_insertion}) mounted within the frame. The mesh should have a wire pitch less than 2 mm to ensure a uniform electric field while maintaining a high optical transparency.


\begin{figure}[t]
  \centering
\includegraphics[width=4in]{figures/TPC_APA_3}
\label{fig:pd_insertion}
  \caption{Photon detectors are mounted within the frame, between the wires on the two sets of four wire layers.  The APA is built so that the photon detectors can be installed through slots in the side of the APA after the APA wires are installed.  The wires that would cross these slots are routed around using copper traces on the edge boards.}
\end{figure}


\subsubsection{CPA and Field Cage}

% input from Bo Yu


Each cathode plane (Fig.~\ref{fig:tpc_cpa_1}) is constructed from 6 identical CPA (cathode plane assembly) modules and two sets of end pieces. Each CPA is about half the size of an APA  (2.3m $\times$ 3.1m) for ease of assembly and transport.  The CPA is made of a stainless-steel framework, 
with 4 pieces of thin FR4 sheets mounted in the openings.  A receptacle for the HV feedthrough is attached to the upper corner of a cathode plane toward the roof entrance side to mate with the HV feedthrough in the cryostat ceiling. 

The FR4 sheets on the CPAs are treated with layers of high resistive coating on both sides.  The resistivity of the coating will be chosen such that the surface potential does not deviate significantly with the ionization current from the cosmic rays, and forms a relatively long time constant to dissipate the stored energy on each sheet in case of a high voltage discharge.  This long RC time constant will also reduce the peak current injected into the front-end electronics in a HV discharge.

Due to the relatively high cosmic ray flux in this surface detector, it is preferable to prevent the scintillation light emitted by a cosmic ray between the cathode and cryostat wall from entering the TPC to reduce false trigger. The opaque cathode surface will service this purpose. The high flux of cosmic rays combined with very low drift velocity of positive ions in the liquid argon will result in sizable space charge distortions in the TPC (docdb \#6471).  In addition, the positive ions could build up further if the ion motion is slowed or stalled by counter flow in the LAr.  Preliminary CFD analysis (docdb \#6140) have shown that solid cathodes in the cryostat result in LAr flow pattern that neither causes excess positive ion buildup, nor degrades the LAr purity.


\begin{figure}[t]
\centering
\includegraphics[width=5in]{figures/TPC_CPA_1}
\caption{Exploded view of a cathode plane constructed from 6 CPA modules and 4 end pieces. The facing material on the CPA is highly resistive to minimize the peak energy transfer in case of a HV breakdown.}
\label{fig:tpc_cpa_1}
\end{figure}

To achieve a 500~V/cm drift field over a 2.5~m distance, the bias 
voltage on the cathode plane must reach $-$125~kV. Two high voltage power supplies (150 -- 200 kV) and two HV feedthroughs will be needed for the two cathode planes.  The HV feedthroughs are based on the Icarus design, but modified to further improve the stability at higher voltages. %(Fig.~\ref{fig:tpc_hv_ft}).

%\begin{figure}[hb]
%\centering
%\includegraphics[width=5in]{figures/TPC_HV_FT}  (`figure not found' -- add it then uncomment)
%\caption{Cross section of the HV feedthrough around the end of the grounded shield, and plot of the equi-potential contours between the HV central conductor and the ground shield. The flared end significantly reduces the electric field at the inside of the shield, improving HV stability.}
%\label{fig:tpc_hv_ft}
%\end{figure}



Each pair of facing cathode and anode rows forms an electron-drift region.  A field cage  completely surrounds the four open sides of this region to provide the necessary boundary conditions to ensure a uniform electric field within, unaffected by the presence of the cryostat walls.

\begin{figure}[htb]
\centering
\includegraphics[width=.49\textwidth]{figures/TPC_FCA_1}
\includegraphics[width=.49\textwidth]{figures/TPC_FCA_2}
\caption{Left: A section of the field cage in the 35ton TPC. Right: Plot of electric field (color contours( and equi-potential contours (black lines) in a small region around the edges of two adjacent field cage strips on a 1.6mm thick FR4 substrate.  A layer of resistive coating between the two copper strips nearly eliminated the high electric field regions at the copper edges }
\label{fig:tpc_fca_1}
\end{figure}    

   

The field cages are constructed using copper-clad FR4 sheets reinforced with fiber glass I-beams to form panels of 2.5~m $\times$ 2.3~m in size for the top and bottom modules, and 2.5~m $\times$ 2~m modules for the sides.  Parallel copper strips are etched or machined on the FR4 sheets.  Strips are biased at appropriate voltages through a resistive divider network. These strips will create a linear electric-potential gradient in the LAr, ensuring a uniform drift field in the TPC's active volume. 
 
Since the field cage completely encloses the TPC drift region on four (of six) sides, with the remaining two sides blocked by the solid cathodes, the FR4 sheets must  be frequently perforated to allow natural convection of the liquid argon.  The ``transparency'' of the perforation will be determined by a detailed LAr computerized fluid dynamic (CFD) study.

The left of Figure~\ref{fig:tpc_fca_1} shows a section of the field cage in the 35ton TPC as it was being assembled.  The 35ton TPC test results will inform us whether we should improve upon the current design, or change the design concept all together for this and future detectors.  The main concern with the current field cage design is that the electric field at the edges of the copper strips is still quite high due to the thinness of the copper.  One possible remedy is to cover the entire surface of the field cage with a high resistive coating.  The resistivity between strips due to this coating must be kept many orders of magnitudes higher than the divider resistance to avoid distortion to the drift field.  Figure~\ref{fig:tpc_fca_1} (Right-Panel) shows an FEA simulation of such a configuration.

In the event of HV discharge on the cathode or the field cage, the voltage differential between neighboring field cage strips near the discharge electrode will be very high for a brief moment.  This over voltage condition could cause damage to the field cage electrode and the resistors installed between strips.  To minimize such disk, varistors or gas discharge tubes (GDT) will be installed between the field cage strips in parallel with the resistors to prevent excess voltage transient between the electrodes. 

In order to test the installation concept of the far detector, the top and bottom field cage modules will be attached to the mating CPAs through hinges.  These combined assembly will be installed into the cryostat and the field cage module opens to bridge the CPA and the APA both mechanically and electrically. 









\subsubsection{TPC Readout}


The TPC front-end (FE) electronics  operate at cryogenic temperatures. 
The system provides amplification, shaping,  digitization, buffering and
multiplexing of the signals. The FE electronics consist of three boards stacked on top of one another:  the Analog Mother Board, the FPGA Mezzanine Board, and the SERDES Mezzanine Board.  Figure \ref{fig:coldelec} shows a schematic of the cold FE electronics. 

\begin{figure}[tb]
  \centering
\includegraphics[scale=0.4]{figures/fe-electronics-block-diagram.pdf}
\label{fig:coldelec}
  \caption{Schematic for the TPC cold FE electronics. ***MG*** Not sure if this is high enough quality.  ***Ask Chen for source and fix numbers.  }
\end{figure}

The Analog Mother Board contains the front-end ASIC chips which perform the analog readout of the TPC wires.  The FE ASIC chip is implemented as a
mixed-signal ASIC providing amplification, shaping, digitization,
buffering, a 16:2 multiplexing stage, a driver and voltage regulators.  The analog-to-digital converter on the ASIC samples each TPC wire at 2 MHz.  Eight such chips 
are mounted on a single readout board, instrumenting 128 adjacent wires in one plane. 

The two (multiplexed) signals from each FE ASIC are fed into the FPGA Mezzanine Board.  The cold FPGA aggregates the TPC data and also supplies the control and clock to the FE ASICs.   The FPGA on the mezzanine board receives the data and packages the 128 channels together, one 2 MHz clock tick at a time.  This is then sent to the SERDES board for serialization and sent to the cryostat flange board over high-speed (1 Gbps) serial links and finally to the DAQ system.  

Besides the high-speed signal cable, which is a twin-axial cable bundle manufactured by GORE, there are cable bundles for low-voltage power, wire-bias voltages, and various slow controls and monitoring. Redundant cables will be provided for many of these functions. The  cable bundles will be connected through a feedthrough on the roof of the cryostat. 


\begin{tabular}{l|c}
\hline
 parameter & value \\
  \hline
 ADC Sampling Rate & 2 MHz\\
More stuff   & \\
 Cluster-on-Boards (COB) & 3\\
Data-Processing-Modules (DPM) & 12 \\
ATCA Shelves  & 1 (6-slot)\\
TPC Readout Compute Nodes & 3 \\
\hline
\end{tabular}



The primary interface between the TPC front-end electronics (FE) and the DAQ subsystem consists of an ATCA-based system of RCEs (Reconfigurable Cluster Elements).  The RCE system receives the serialized raw data for the FE, performs zero-suppression on it, and packetizes and transmits the resulting sparsified data to a back-end data farm for event building and further processing.  Additionally, the RCE system transmits timing and control signals to the FE as well as forwarding configuration data to them at start-up.     

The RCE system consists the following components:  a commercial ATCA shelf (2-, 6-, or 14-slot), a Cluster-On-Board (COB) which is the "front board" in ATCA terms, and a Rear-Transition-Module (RTM) which is the "rear board". A schematic of the system is shown in Figure \ref{fig:rceblock}.  The COB is a custom board, developed by SLAC, which holds the processing power of the system.  The COB (see Figure \ref{cob}) consists of 5 bays for holding daughter boards, an onboard 10-GbE switch, and both 10- and 1-Gb ethernet connections for communications with the back-end system.  Four of the daughter-board bays are for Data Processing Modules (DPM), each of which can hold up to two RCEs.  The RCE is the core procession unit of the system; it is made up of a modern SoC (currently, the Xilinx Zynq-7045) with multiple high-speed I/O ports (up to 10-Gbps each) and external DRAM and flash memory controllers.  The other bay on the COB contains the Data Transmission Module (DTM) which is responsible  for distributing timing and trigger information to and between the DPMs.  

While the COB hardware is application agnostic,  the RTM is application specific. The RTM provides the mechanical interface between the front-end (or, in our case, the flange electronics) and the back-end, as well as other external sources such as the timing or trigger systems.  In this case we will use fiber optic connections between the flange and the TPC DAQ using 8 12-channel (full duplex) CXP connectors on the RTM. 

With the assumption that each cold FE board multiplexes it's 128 wire channels to 4 outputs at 1-Gbps each, the non-zero suppressed data for 1 APA can be fed into a single COB (containing 8 RCEs).  Each RCE would receive data from 2 FE boards, perform zero-suppression, and send the result to the back-end.  

MG***data rates?  

\begin{figure}[tb]
  \centering
%\includegraphics[scale=0.4]{figures/cern-tpc-daq-block.pdf}
\includegraphics[scale=0.4]{figures/rce-block.pdf}
\label{fig:rceblock}
  \caption{Schematic for the TPC DAQ system.   }
\end{figure}


\begin{figure}[hbt]
  \includegraphics[scale=0.6]{figures/COB-gen3.pdf}
  \caption{\label{fig:cob} The COB (left of the large connectors) and RTM (right).  }
\end{figure}






\subsubsection{Photon Detection System}
\subsection{Photon Detector}
This section to provide context and illustrate which aspects need testing at the cERN prototype

\subsection{Introduction}



\subsubsection{DAQ, Slow control and monitoring}
The DAQ will merge data to form events from the LArTPC, 
photon detector and beam detector readouts using the 
artDAQ data acquisition toolkit using a farm of commercial 
computers connected with an Ethernet switch.  ArtDAQ is 
in use on several experiments at Fermilab.  We are using it
on the 35t prototype, so we will have considerable 
experience by the time of the CERN test.  

The data collection for the CERN test will operate in a mode 
similar to that forseen for the underground detectors. In order 
to collect data from non-beam interactions such as proton decay 
candidates or atmospheric neutrinos, data will be continuously
read in to the artDAQ data receiver nodes and processed through
the artDAQ system in quanta corresponding to time intervals fixed
from the time of the beginning of the run.  These are then 
transferred through the switch to a set of event building nodes 
which work in parallel, each node receiving all the data from all 
the detectors for the time intervals it is responsible for processing.
There will be 32 parallel incoming data streams from the LArTPCs
and 16 streams from the photon detectors.  

	There will be an additional stream from the trigger board (the same
board as built by Penn for the 35t test will be used) which will receive input
of the spill gate, warning of extraction, and pattern-unit bits from trigger
counters and other beamline instrumentation such as Cerenkov counters [Which
section are these described in?, should we refer to them from here?]. The
existing trigger board takes as input logical signals from the various sources,
and provides both a hardware trigger signal as well as a trigger word
specifying which triggers are active in any given event (beam spill, cosmic
veto, calibration trigger, random trigger, etc.).  The trigger board is built
around a MicroZED development board carrying a Xilinx Zynq 7020 and is
therefore highly configurable, both through firmware (``process logic'') side
and software (``process system'') side.  With small hardware modifications, the
board could take analog inputs if, for example, the signals from beam
instrumentation or veto counters are provided in a raw form from their
front-end.

Synchronisation across all the input sources is essential in order 
that artDAQ can bring together the data from the input streams correctly for
processing by the event building nodes.  The data receiver nodes will provide
overlap by repeating the data at the boundaries of the time intervals so 
that a particle whose data spans two time intervals can be collected.  
The time synchronisation is provided to the RTM back-module on the LArTPC 
readout crates, to the SSP photon detector readout and to the trigger board from
a GPS based time synchrononisation distribution system originally designed 
for the NOvA experiment.  This system includes functionality to calibrate and 
correct for the cable delays, and to send synchronisation control signals to
the readout at predetermined times.

The event building nodes will select time regions of interest within the time 
intervals they are processing and form these into events to be written to 
disk. The algorithms to select the events may be as simple as looking for 
a trigger bit in the trigger board data stream, or may involve looking 
for self-triggered events in the LArTPC data.  An aggregation task, which 
is part of artDAQ will handle the parallelized event building processes by 
merging the output events into a single stream and writing them to disk.
To avoid oversized output data files, when a predetrmined file size is reached, 
the aggregator will switch to writing to a new file.  The collaboraion 
requests to CERN, data links of sufficient bandwidth to transfer these files 
from the CENF to the CERN data center, and from there to locations 
worldwide for analysis. 

Improved versions of the software systems which are being prototyped at the 
35t test will available for the CERN test including (a) Run control which 
controls and monitors the DAQ processes and allows run starts and stops to
be performed by the operator (b) online monitoring (c) slow control of 
voltages and temperatures being used by the electronics (this may not be 
comprehensive by the time of the CERN protoype, but we plan on prototyping 
the readout of some of the quantities).  The trigger board includes facilities 
for generating calibration pulses and for identifying the event times of 
the calibration events.



\subsubsection{Installation}

The interior of the cryostat will be prepared prior to the installation of the TPC.  A series of support rails 
will be suspended below the top surface of the cryostat membrane.  These will be structurally supported 
by a truss structure above the cryostat.  These supports will pass through the top of the cryostat.  They 
need to be designed to minimize the heat gain into the cryogenic volume.  For the CPAs, the rails need to 
be electrically isolated due to high voltage concerns.  To preserve the ability to reverse the order of the 
TPC components, all of the support rails will be designed to the same set of requirements regarding 
loads and attachment points.  

There will be a series of feed thru flanges located along each of the support rails.  These will be cryogenic 
flanges where the services for the TPC components can pass through the top of the cryostat.  It is 
foreseen that each CPA will require one feed thru for the high voltage probe to bring in the drift voltage.  
The drift voltage is 500 V/cm.  For a drift distance of 2.5 m, the probe voltage will be 125 kV.  There will 
be one service feed thru for each of the APAs.  These feed thrus will include high speed data, bias 
voltages for the wire planes, control and power for the cold electronics.  

The main TPC components will be installed through a large hatch in the top of the cryostat.  This is 
similar to the installation method intended for the detector at the far site.  This hatch will have an 
aperture approximately 2.0 m wide and 3.5 m long.  Each APA and CPA panel will be carefully tested after 
transport into the clean area and before installation into one of the cryostats. Immediately after a panel is 
installed it will be rechecked. The serial installation of the APAs along the rails means that removing and 
replacing one of the early panels in the row after others are installed would be very costly in effort and 
time. Therefore, to minimize the risk of damage, as much work around already installed panels as 
possible will be completed before proceeding with further panels.
The installation sequence is planned to proceed as follows:

\begin{enumerate}
\item Install the monorail or crane in the staging area outside the cryostat, near the equipment hatch.
\item Install the relay racks on the top of the cryostat and load with the DAQ and power supply crates.
\item Dress cables from the DAQ on the top of the cryostat to remote racks.
\item Construct the clean-room enclosure outside the cryostat hatch.
\item Install the raised-panel floor inside the cryostat. 
\item Insert and assemble the stair tower and scaffolding in the cryostat.
\item Install the staging platform at the hatch entrance into the cryostat.
\item Install protection on (or remove) existing cryogenics instrumentation in the cryostat.
\item Install the cryostat feedthroughs and dress cables inside the cryostat along the support beams.
\item Install TPC panels:
   \begin{enumerate}
   \item Install both CPA panels.  These will be installed from the floor of the cryostat.  Access to the top edge will be required by scaffolding.  
   \item Install and connect HV probe for each of the CPAs.
   \item Perform electrical tests on the connectivity of the probe to the CPAs.
   \item Install first end wall of vertical field cage at the non-access end of the cryostat.  These will be installed from the floor of the cryostat.  Scaffolding will be needed to install the supporting structure and then attach the panels to the structure.  
   \item Test the inner connections of the field cage panels.
   \item Install the first APA and connect to the far end field cage support.
   \item Connect power and signal cables.  This will require scaffolding to access the top edge of the APA.
   \item Test each APA wire for expected electronics noise. Spot-check electronics noise while cryogenics equipment is operating.
   \item Install the upper field cage panels for the first APA between the APA and CPAs  This will require scaffolding to access the upper edge of the APA, CPA and field cage structure.
   \item Perform electrical tests on upper field cage panels.
   \item Repeat steps (f) through (j) for the next two APAs.
   \item Install the lower field cage panels between the APAs and CPAs.  Start at the far end away from the access hatch and work towards the hatch. 
   \item Perform electrical test on lower field cage panels and the entire loop around the TPC.
   \item Remove temporary floor sections as the TPC installation progresses.
   \item Install sections of argon-distribution piping as the TPC installation progresses.
   \item Install the final end wall of vertical field cage at the access end of the cryostat.  These will be installed from the floor of the cryostat.  Scaffolding will be needed to install the supporting structure and then attach the panels to the structure.
   \end{enumerate}
\item Remove movable scaffold and stair towers.
\item Temporarily seal the cryostat and test all channels for expected electronics noise.
\item Seal the access hatch.
\item Perform final test of all channels for expected electronics noise.
\end{enumerate}
 
In general, APA panels will be installed in order starting with the panel furthest from the hatch side of the cryostat and progressing back towards the hatch. The upper field cage will be installed in stages as the installation of APA and CPAs progresses.  After the APAs are attached to the support rods the electrical connections will be made to electrical cables that were already dressed to the support beams and electrical testing will begin. Periodic electrical testing will continue to assure that nothing gets  damaged during the additional work around the installed APAs.  

The TPC installation will be performed in three stages, each in a separate location; the locations, or 
zones. First, in the clean room vestibule, a crew will move the APA and CPA panels from storage racks, rotate to the vertical position and move them into the cryostat. Secondly, in the panel-staging area immediately below the equipment hatch of the cryostat, a second crew will transfer the lower panels from the crane to the staging platform, connect the upper and lower panels together, route cables to the top of stacked panels and finally transfer the stacked panels on to the rails within the cryostat. A third crew will reposition the movable scaffolding and use the scaffold to make the mechanical and electrical  connections at the top for each APA and CPA as they are moved into position.  

The requirements for alignment and survey of the TPC are under development. Since there are many cosmic rays in the surface detector and beam events, significant corrections can be made for any misalignment of the TPC. The current plan includes using a laser guide or optical transit and the adjustment features of the support rods for the TPC to align the top edges of the APAs in the TPC to be straight, level and parallel within a few mm. The alignment of the TPC in other dimensions will depend on the internal connecting features of the TPC.  The timing of the survey will depend on understanding when during the installation process the hanging TPC elements are in a dimensionally stable state. The required accuracy of the survey is not expected to be more precise than a few mm.  




