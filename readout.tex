




\begin{figure}[htb]
\centering
\begin{minipage}[b]{1.0\textwidth}
\begin{center}
\includegraphics[width=.75\textwidth]{figures/fe-electronics-block-diagram.pdf}
\end{center}
\end{minipage}
\caption{Layout of the TPC cold from end (FE) electronics..}
\label{fig:coldelec}
\end{figure}



The electronics for the TPC is designed to operate at liquid argon temperature and is places as close to the sense wires as possible. By minimizing the capacitance loading the preamplifiers the electronics noise is greatly reduced. The present design has a maximum wire length of 7.3~m (induction planes) with corresponding capacitance of 164~pF and expected intrinsic noise of 400 electrons. In order to minimize any coupled noise and to reduce the cable count a 12 bit cryogenic ADC operating at 2 MS/s has also been designed which includes a 1:8 multiplexing stage. The output of the ADCs are then read out by a commercial  FPGA. The FPGA receives the data and is capable of providing an additional factor of 4 in multiplexing if no zero suppression is applied.  This data is then sent out of the cryostat using the FPGAs high-speed (1 Gbps) serial links and finally to the DAQ system.  For the final detector it is expected that a dedicated digital control and data transmission asic will be developed which replaces the commercial FPGA but this work is only now starting and it will not be available at the start of the CERN test. The front end electronics is organized as a stack of three boards:  the Analog Mother Board with the preamplifiers and ADCs that mounts on the APA, the FPGA Mezzanine Board, and a SERDES Mezzanine Board serving as a cable interface.  Each analog mother board has eight preamplifiers/ADCs and instruments 128 wires. A Faraday cage covers the end of the APAs to shield the electronics and also to prevent any bubbles formed on the components from entering the active TPC volume. Figure \ref{fig:coldelec} shows a schematic of the cold electronics. 

Besides the high-speed signal cable, which is a twin-axial cable bundle manufactured by GORE, there are cable bundles for low-voltage power, wire-bias voltages, and various slow controls and monitoring. Redundant cables will be provided for many of these functions. The  cable bundles will be connected through a feedthrough on the roof of the cryostat. 


\begin{figure}[htb]
\centering
\begin{minipage}[b]{1.0\textwidth}
\begin{center}
\includegraphics[width=1.0\textwidth]{figures/rce-block.pdf}
\includegraphics[width=.5\textwidth]{figures/COB-gen3.pdf}
\end{center}
\end{minipage}
\caption{Top: Schematic for the TPC DAQ system. Bottom: The COB (left of the large connectors) and RTM (right).}
\label{fig:rce}
\end{figure}

The primary interface between the TPC front-end electronics (FE) and the DAQ subsystem consists of an ATCA-based system of RCEs (Reconfigurable Cluster Elements).  The RCE system receives the serialized raw data for the FE, performs zero-suppression on it, and packetizes and transmits the resulting sparsified data to a back-end data farm for event building and further processing.  Additionally, the RCE system transmits timing and control signals to the FE as well as forwarding configuration data to them at start-up.     

The RCE system consists the following components:  a commercial ATCA shelf (2-, 6-, or 14-slot), a Cluster-On-Board (COB) which is the "front board" in ATCA terms, and a Rear-Transition-Module (RTM) which is the "rear board". A schematic of the system is shown in Figure \ref{fig:rce}.  The COB is a custom board, developed by SLAC, which holds the processing power of the system.  The COB (see Figure \ref{fig:rce}) consists of 5 bays for holding daughter boards, an onboard 10-GbE switch, and both 10- and 1-Gb ethernet connections for communications with the back-end system.  Four of the daughter-board bays are for Data Processing Modules (DPM), each of which can hold up to two RCEs.  The RCE is the core procession unit of the system; it is made up of a modern SoC (currently, the Xilinx Zynq-7045) with multiple high-speed I/O ports (up to 10-Gbps each) and external DRAM and flash memory controllers.  The other bay on the COB contains the Data Transmission Module (DTM) which is responsible  for distributing timing and trigger information to and between the DPMs.  

While the COB hardware is application agnostic,  the RTM is application specific. The RTM provides the mechanical interface between the front-end (or, in our case, the flange electronics) and the back-end, as well as other external sources such as the timing or trigger systems.  In this case we will use fiber optic connections between the flange and the TPC DAQ using 8 12-channel (full duplex) CXP connectors on the RTM. 

With the assumption that each cold FE board multiplexes it's 128 wire channels to 4 outputs at 1-Gbps each, the non-zero suppressed data for 1 APA can be fed into a single COB (containing 8 RCEs).  Each RCE would receive data from 2 FE boards, perform zero-suppression, and send the result to the back-end.  








